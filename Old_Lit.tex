
\section{Literature \label{Literature}}

\subsection{High-Frequency Trading}

Following the academic literature, we apply the label high-frequency trading (HFT) to low-latency algorithmic trading (orders generated via computer algorithms) that typically involve some combination of high total daily trading volume, bounded end-of-day or intra-day positions, and frequent changing between net short and long positions \cite[e.g.]{Kirilenko2015, Clark-Joseph2013}. Some authors also emphasize high messaging rates (orders vs. trades), holding/cancellation periods and number of messages following defined events \cite[e.g.]{Malinova2014}.

Most academic papers studying HFT use proprietary data with trader identification, and are able to classify accounts as either aggressive (primarily liquidity consuming) or passive (primarily liquidity providing). Passive accounts are almost uniformly associated with positive market impact. \cite{Jovanovic2015} find that passive HFT reduces spreads by 15\%, \cite{Hagstromer2013} find that passive HFT reduces short-term price volatility and \cite{Menkveld2015b} show that high-frequency market making increases price efficiency at the moment of incoming news, which reduces adverse selection costs and lowers the bid/offer spread. \cite{Malinova2014} show that a fee increase targeting high-frequency market-making activities on six Canadian exchanges reduced liquidity and increased spreads, and \cite{Brogaard2015b} uses Canadian equities data to show that HFT (market-making) limit orders are associated with 60-80\% of price discovery and have a 50\% larger price impact than non-HFT limit orders.

The effects of aggressive HFT, however, are mixed; it is generally associated with informed price impact, especially over short horizons. Aggressive HFT can increase adverse selection costs for other traders, increase short-term volatility, and raise trading costs for institutional and retail traders, as shown by \cite{Brogaard2013, Zhang2011, Menkveld2015b}. \cite{Hendershott2013} show that aggressive HFT is more prevalent when spreads are narrow. When aggressive and passive HFT are aggregated together, estimated effects are often positive, but usually with the acknowledgement of associated costs, e.g., \cite{Brogaard2015a}, \cite{Hasbrouck2013}, \cite{Bershova2012}, and \cite{Breckenfelder2013}. The findings in \cite{Hirschey2013} suggest that HFT behavior provides a net improvement to liquidity, but increases costs for non-HFT traders. \cite{Baldauf2015a} show that adverse selection costs associated with HFT outweigh the benefits of smaller bid/ask spreads in fragmented CDA markets. Additionally, \cite{Baldauf2015} show that the benefits of lower transactions costs in the presence of HFT are offset by less informative prices. In these latter papers the authors consider alternative market designs, and predict improved market efficiency under a selective order delay format and under FBA.

Two limitations of empirical work should be noted. First, all data come from minor variants of CDA, and offer only indirect evidence about the costs and benefits of other formats. Second, aside from a few scattered episodes such as the Flash Crash [\cite{Aldrich2016}; \cite{Kirilenko2015}; \cite{Menkveld2015}], little is known about the impact of HFT in times of financial stress as opposed to normal times. The laboratory is an ideal environment to study outcomes under new market formats and to examine market reactions to events that are extreme and unlikely in observational data.

\subsection{Financial Market Design and Lab Studies}

The broad field of market design studies how different market institutions (a.k.a. market formats) operate in the real world, both empirically and theoretically. It focuses on what features of these institutions help or hurt efficiency, revenue, stability, etc. Over the last two decades, many markets have been “designed” by economists, from radio spectrum licenses to Treasury bills (see \cite{Klemperer2004} and \cite{Milgrom2011} for surveys of auction market design).

The specific literature on financial market design, where our paper lies, studies the theory and empirics of the performance of market features and formats of financial market formats. This literature has utilized different approaches, including modeling, simulations, econometrics on observational data and, importantly, laboratory experiments. Besides those mentioned in the subsection on HFT research, other important contributions in this literature include \cite{Roth1994} who study the timing of transactions and \cite{Roth1997} who study serial versus batch processing; \cite{Foucault1999} and \cite{Roth2002} who introduce the idea of bid sniping; \cite{Du2017} and \cite{Fricke} who study the optimal frequency of double auctions, and \cite{Biais2014a} who study “fast trading” and the externalities it generates.

Laboratory studies of financial markets date back to the beginnings of experimental economics. As mentioned earlier, appeal to studying market features and formats is that in the lab simplified settings allow researchers to make precise head-to-head comparisons of market formats or features. This, in turn, uncovers the forces or incentives behind the evolution of prices and trading volume. The early laboratory market study include \cite{Smith1962} which featured versions of the CDA and found efficient outcomes within a few periods with only a few buyers and a few sellers and \cite{Plott1978} first compared two market formats in the lab, and found that the posted-offer was slightly less efficient than the CDA. Thousands of market experiments followed, including many focusing on financial markets. Leading topics include (a) the capacity of different market formats to aggregate information [e.g. \cite{Plott1982}; \cite{Plott1988}; \cite{Forsythe1990}; \cite{Copeland1987}; \cite{Copeland1991}], (b) the convergence to the rational expectation equilibrium [e.g. \cite{Forsythe1982}; \cite{Forsythe1984}; \cite{Friedman1984}], (c) price bubbles [e.g. \cite{Smith1988}; \cite{Noussair2006}], and (d) information mirages [see e.g. \cite{Camerer1991}]. For surveys in experimental research in financial markets see \cite{Holt1995}, \cite{Sunder1995}, \cite{Friedman2008}, and \cite{Noussair2013}. \cite{Friedman1993} is a standard reference on financial market formats. It reports on the first open tournament (for perishables) using a variant of the CDA. To our knowledge, none of the laboratory studies so far focus on market-making (although it is a secondary theme in a few studies such as \cite{Friedman1993}) or consider noise traders as the primary source of profits.

Comparisons of CDA to FBA performance include \cite{Cason1996}, \cite{Cason1997}, \cite{Cason1999}, \cite{Cason2008}, who find that both formats are quite efficient in laboratory markets for perishables with independent private values. Although \cite{Huber2008} study whether or not traders' profits in CDA and FBA are monotonic with respect to individual information, at present little is known about comparative performance in common value asset markets, which is most relevant for HFT.

The work of \cite{Budish2015} (henceforth BCS) is especially relevant as it provides the theoretical framework and predictions to our paper. A summary of the model and predictions are provided in later in Environment subsection.

